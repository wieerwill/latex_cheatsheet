\documentclass[a4paper]{article}
\usepackage[ngerman]{babel}
\usepackage{cheatsheet}

\cheatsheettitle{Cheatsheet Example}
\cheatsheetauthor{Robert Jeutter}

\begin{document}

\begin{multicols}{3}\scriptsize

    \section{Section}
    \subsection{Subsection}
    \subsubsection{Subsubsection}
    \paragraph{Paragraph}
    Some random text to show minimum space required. This is only an example, everybody who reads this does it to his own harm. We can write here a lot of stuff but it really is only for demonstration.

    \section{Lists}
    % use * to reduce required space
    Item lists
    \begin{itemize*}
        \item This is
        \item a small list
        \item to show some
        \item space
    \end{itemize*}

    Enumeration Lists
    \begin{enumerate*}
        \item This is
        \item a small list
        \item to show some
        \item space
    \end{enumerate*}

    Description Lists
    \begin{enumerate*}
        \item[This] is a
        \item[Small] list to
        \item[Show] some space
    \end{enumerate*}

    \columnbreak

    \section{Boxes}

    \note{Title}{This is a little but helpful note you can use to highlight informations}

    \note{}{Not all notes need captions.}

    \warning{Warning}{This is a warning box.}

    \warning{}{This is a another warning.}

    \important{Important}{This might be something important to remember!}

    \important{}{Some important boxes don't need caption}

    \headnote{Notes can have titles}{And a text below.}

    \headnote{}{This note has no title}

    \newtcbox{\mynewbox}[1][red]{on line,
        arc=0pt,outer arc=0pt,colback=#1!10!white,colframe=#1!50!black,
        boxsep=0pt,left=1pt,right=1pt,top=2pt,bottom=2pt,
        boxrule=0pt,bottomrule=1pt,toprule=1pt}
    \newtcbox{\xmybox}[1][red]{on line,
        arc=7pt,colback=#1!10!white,colframe=#1!50!black,
        before upper={\rule[-3pt]{0pt}{10pt}},boxrule=1pt,
        boxsep=0pt,left=6pt,right=6pt,top=2pt,bottom=2pt}

    Need to \lhigh{highlight} some \lhigh[green]{words} in your text? \rhigh{Now} possible in \rhigh[blue]{two different} styles.

    \columnbreak

    \section{Code}
    \begin{lstlisting}
    main(void){
        for(int i=0; i<10; i++ ){
            //do something here 
            x+= 2;
        }
        echo("The loop is finished")
    }
    \end{lstlisting}


    \section{Tables}
    \begin{tabular}{l|c|r}
        Nr & Sign     & Code              \\\hline
        1  & $\Omega$ & $\backslash$Omega \\
        2  & $\Theta$ & $\backslash$Theta \\
        3  & $\rho$   & $\backslash$rho
    \end{tabular}

    \begin{tabular}{L{0.1\linewidth}|R{0.8\linewidth}}\scriptsize
        84   & \href{http://www.gutenberg.org/ebooks/84}{Frankenstein; Or, The Modern Prometheus by Mary Wollstonecraft Shelley} \\
        6087 & \href{https://www.gutenberg.org/ebooks/6087}{The Vampyre; a Tale by John William Polidori}                        \\
        696  & \href{https://www.gutenberg.org/ebooks/696}{The Castle of Otranto by Horace Walpole}                              \\
        42   & \href{https://www.gutenberg.org/ebooks/42}{The Strange Case of Dr. Jekyll and Mr. Hyde by Robert Louis Stevenson}
    \end{tabular}

\end{multicols}


\flashcard{Definition}{Your Question?}{Maybe a theme?}{This is the answer to your Question}
\flashcard{}{Minimal Example}{}{Here is very little text}

\flashcard{}{Print a List}{This could be a hint}{You can fully configure the back:
    \begin{itemize}
        \item Test
        \item Two
        \item Three
    \end{itemize}
}

\end{document}